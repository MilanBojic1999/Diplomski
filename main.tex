\documentclass[12pt, letterpaper, oneside]{article}
\usepackage[utf8]{inputenc}
\usepackage[LGR,T1]{fontenc}
\usepackage{braket}
\usepackage{amsmath}
\usepackage{biblatex}
\usepackage{bbold}
\usepackage{quantikz}

\addbibresource{refx.bib}
\newtheorem{definition}{Definition}

\title{Kvantno masinsko ucenje}
\author{Milan Bojic}
\date{Jun 2022}

\renewcommand*\contentsname{Sadrzaj}

\begin{document}

\maketitle
\newpage

\tableofcontents
\newpage
\section{Uvod}
Prethodna decenija je bila obeležena mašinskim učenjem, sa njenom primenom u svakodnevnom životu običnih ljudi.
U prethodnih nekoliko godine počelo je da se oseća usporenje inovacija i razvoja novih metoda, u nekim oblastima dostigle su se granice računarskih resursa (npr. GPT-3).
U rešeavanju ovog problema očekuje se da pomogne razvoj kvantnog računarstva, oblast koja koja se razvoja gotovo pola veka, ali tek u poslednjih nekoliko godina je došlo do
povećanog interesovanja. Za kvantno računarstvo se očekuje da bude sledeće veliko remećenje tehnološkog poretka, sa približavanjem kvantnoj superiornostu svakog dana. \\
Od kvantno mašinsko učenje da bude prekretnica u mašinskom učenju kakvog danas znamo. Oblast je relativno mlada i trenutno se najviše bavi teoretskim razvojem.
Neki od zadataka kojim će se baviti jeste obrada kvantnih sistema i "učenje" njihovih osobina, brže i bolje prepoznavanje obrazaca u klasičnom sistemu, kao i
otkrivanje nekih osobina klasičnog mašinskog učenja koji se nisu mogli primetiti u klasičnom sistemu. \\
U ovom radu predstaviću nekoliko metoda koji su bila istraživana u prethodnom periodu. 
\section{Kvantno racunarstvo}
Pre nego sto se pocne pricati o Kvantnom masinskom ucenju, treba objasniti neki osnovni pojmovi da bi lakse razumeli ostatak rada

\subsection{Osnovni pojmovi}
Potrebni pojmovi su:
\begin{itemize}
    \item Kubit (eng. Qubit)
    \item Kvantna kola (eng. Quantum Gates)
    \item Kvantna uvezanost (eng. Quantum entanglement)
    \item Kvantan memorija, Kvantni registri
\end{itemize}

\subsubsection*{Kubit}
Kubit (eng. Qubit ) je najmanja jedinica informacije u kvantnom računarstvu, slično bit-u u klasičnom računarstvu.
Razlika od bita jeste u tome što kubit pored stanja 1 i 0, može da se nalazi i u superpoziciji između oba.
Oni se mogu predstaviti formulom (koristeci "bra-ket" notaciju):
\[ \ket{\gamma} =  \alpha\ket{0}+\beta\ket{1} \]
Ovde su $\ket{0}$ i $\ket{1}$ zapravo stanja kao i kod klasičnog bita, a $\alpha$ i $\beta$ su kompleksni brojevi koji predstavljaju aplitude zadatih stanja i za njih važi:

\[ |\alpha|^2+|\beta|^2 = 1 \]
Posto stanje kubita ima dva stepena slobode sto dovodi do toga da amplitude se mogu zapisati kao:
\[
    \alpha = \cos{\frac{\Theta}{2}} 
\]
\[
    \beta = e^{i\phi}\sin{\frac{\Theta}{2}}
\]
Takodje mozemo da vidimo da je $|\alpha|^2$ verovatnoca da se kubit nalazi u stanju 0, isto vazi i za $ |\beta|^2$ i 1.
Saznanje o tomo u kom stanju se nalazi kubit ce se dobiti merenjem kubita, tade ce da kubit izadje iz superpozicije i "pasce" u stanje 1 ili stanje 0. U tom slucaku kubit ce imati ponasanje kao i obican bit, ali ovako gubimo predjasnje kvantno stanje kubita.
U fizičkom svetu kubit se moze predstaviti kao polarizovani fotoni, pre cemu se dva stanje se uzimaju kao vertikalna i horizontalna polarizacija.
\subsubsection*{Kvantna kapija}
Kvantna kapije (eng. Quantum Gates ) su logički predstavljaju matricama i oni rade na određenom broju kubita.Matrice su unitarne sa oblikom $2^n \times 2^n$, gde je $n$ broj qubita na kojim radimo . Neke od poznatih kola su: Hademardovo kolo (stavalja kubit u superpoziciju), bit flip kolo (zamenjuje aplitude na kubitu), ali nas najviše zanima rotaciono kolo:
\[
    R = \begin{bmatrix}
        \cos{\Theta} & -\sin{\Theta} \\
        \sin{\Theta} & \cos{\Theta} 
    \end{bmatrix}
\]
Ovo kolo rotira kubite u prostoru, odnosno menja njihove amplitude za $\Theta$ radiana.
%% Pogledati Quantum Computation and Quantum Information ch. 1.3 Quantum computation za dopune
\subsubsection*{Kvantna uvezanost}
Kvantna uvezanost (eng. Quantum entanglement) je fizički pojam gde su dva ,ili više, kubita povezana tako da zajedno prave novo kvantno stanje.
U čistim stanjima oni su matematički zapravo proizvodi tenzora amplituda:
\[
    \ket{\gamma} \otimes \ket{\delta} = \alpha_1\alpha_2\ket{00} + \alpha_1\beta_2\ket{01} + \beta_1\alpha_2\ket{10} + \beta_1\beta_2\ket{11}
\]
I ovako napisano kvantno stanje se moze razdvojiti na dva kubita. Ali postoje i kvanta stanja koja se ne mogu razdvojiti npr.
\[
 \frac{1}{\sqrt{2}}\ket{00} + \frac{1}{\sqrt{2}}\ket{11}
\]
Zanimljiva stvar kod uvazanih kubita jeste u tome što dele informacije. Ako bi jedan kubit iz para odneli u neko veoma daleko mesto (na primer druga galaksija), i tamo bi ga izmerili mi bi smo dobili 0 ili 1, međutim drugi kubit bi takođe upao u određeno stanje i to u istom trenutni kad smo izmerili prvi daleki kubit. Ovo je zapravno gde se nalazi glavan različitost između klasičnog i kvantnog računarstva, ova pojava ne postoji u klasičnom računarstvu i ne može se "lako" simulirati.
\subsubsection*{Kvantni registri}
Kvantni registri se sastoje od kvantnog stanja od $m$ uvezanih kubita i moze da se predstavlja do $2^m$ vrednosti stanja istovremeno.
Kvantan memorija su uredjaji koji cuvaju kvantna stanja fotona, bez da unistavaju kvanten informacije koja se nalazi u fotonu.
Ovakva memorija zahteva koherentni sistem materije, jer bi u suprotnom kvantna informacija unitar uredjaja bila izgubljena zbog nekoherentnosti.

\subsection{Kvantno racunarstvo}
Kvantno racunarstvo je vrsta racunarstva gde se koriste kolekcije fizickih osobina kvantne mehanike kao sto su superpozicija i kvantna uvezanost,
tako da se izvrsi neka kalkulacija. Uredjaja koji izvrsava kvante kalkulacije zovu se \textbf{kvantni racunari}.
Kvantni racunari se sastoje od kvantnih kola i elemntarnih kvantnih kapija koje sluze za prenosenje i manipulisanje kvantnih inforamcija.
\cite{nielsen_chuang_10th} \\
Jedna od glavnih primena Kvantnih racunara jeste simulacija fizickih sistema, bili oni kvantne ili klasince prirode.
%% Pogledati Quantum Computation and Quantum Information za dopune
\subsection{Kvantna inforamcija}
Kvantan informacija je informacija o stanju kvantnog sistema. O njihovim svojstvima bavi se \textbf{kvantna teorija informacije}.
Takodje, kvantan informacija mogu izmeriti na isti nacin kao i klasicna informacija koristeci se Šenononvoj metodi. 
Odnosno, postoji jednistveno merilo to jest funkcija nad kvantnim stanjem, koje je funkcija verovatnoce, kontinuiteta i sumiranja.\cite{vlatko_v}
Ova funkcija se zove \textbf{von Neumann entropy} i za neki ulazni kubit $\varrho$ postoji ekvivalent u  \textbf{Shannon entropy} $H$
za neku slucajnu promenljivu $X$
\[
    S(\varrho) = H(X)
\]
%% Opisati metod u Apendexu mozda
Jos jedna od merila za kvantno stanje jeste merenje "validnost" (eng. fidelity) izmedju dva kvantna stanja $ \ket{\phi}$ i $\ket{\psi}$.
Neka je $F$ funkcija koja meri osobinu, ona meri verovatnocu da merenjem stanja $\ket{\phi}$ dobijemo stanje $\ket{\psi}$.
Izlaz funkcije je izmedju 0 i 1, gde ako je izlaz 0 onda su dva stanja ortogonalna jedna od drugog, a ako je izlaz 1 onda su dva stanja jednaka.\cite{vlatko_v}
\subsubsection*{Odnost kvantne i klasicne teorije informacije}
Kvantna i klasicna informacija se u dosta stvari razlikuju. Dok se klsaicna informacija prolazi kroz sisteme sa dobro definisanim stanjima, moze se kopirati i pri procesom merenje se nemenja,
Kvantna informacija je enkodovana u kvantnim sistemima, ne moze se kopirati i pri procesu merenje ona se menja. Takodje kvantan informacija ima neke osobine koje se ne
mogu iskazati u klasicnoj informaciji, kao sto su superpozicija i kvantan uvezanost \cite{Classical&quantum_info} \\
Kvantana teorija informacije se bavi: 
\begin{enumerate}
    \item Prenosenju klasicnih informacija preko kvantnih kanala
    \item Prenosenje kvantinih informacija preko kvantinh kanala
    \item Efekat kvantne uvezanosti na prenosenje informacija
\end{enumerate}
\subsubsection{Priprema podataka}
Za obradu podataka treba nam kvantni RAM (QRAM), koji nam dozvoljava paralelan pristup kvantnim podacima. 
Neka imamo kopleksan vektor $\overrightarrow{v}$ sa $N=2^n$ dimenzija, gde su njegove komponente oblika 
\[
    v_j = |v'_j|e^{i\varPhi_j}
\]

Ako imamo parove ${|v'_j|,\varPhi_j}$ cuvamo kao float brojeve u QRAM-u, onda mozemo da konstruisemo
$\log_{2}N$ kubit kvantno stanje $\ket{v} = |\overrightarrow{v}|^{-\frac{1}{2}}\overrightarrow{v}$ u $O(\log N)$ koraka

Kada smo kreirali kompresovane kvantne vektore od ulaznih vektora, mozemo da vrsimo transformacije koristeci kvantne algoritme, za dalje koriscenje podataka za masinsko ucenje.
Ovaj proces zove se \textbf{postprocessing} i u opstem obliku njemu je potrebno $O(poly(\log{}N))$ koraka. \cite{lloyd2013quantum}


\section{Linearne algebra za kvantno masinsko ucenje}
Da bi videli kako kvantni racunari poboljsavaju masinsko ucenje, treba da se vidi kako kvantni racunari obradjuju linearnu algebru, jednu od osnova modernog masinskog ucenja. \\
Tokom godina razvijeni su nekoliko kvantnih algoritama koji resavaju probleme linearne algebre.
Zajedno ti algoritmi se nazivaju \textbf{osnovni kvantni podprogrami linearne algebre} (eng. qBLAS), i oni se koriste u izradi algoritama za kvantno masinsko ucenje.

Primeri algoritama koji su deo qBLAS-a su: 
\begin{itemize}
    \item HHL algoritam: koristi se za resavanje sistema linearnih jednacina, koristeci $2^n$ dimenzijonalni vektorski prostor
        za resavanje sistema sa $n$ promenljivih. \cite{Quantum_machine_learning}
    \item Kvantna Furijeova transformacije \cite{Classical&quantum_info_Fourie_Phase}
    \item Kvantan procena faza za eigen vrednosti i eigen vektora/stanja. \cite{Classical&quantum_info_Fourie_Phase}
\end{itemize}
Ovi algoritmi se koriste kao osnova napredinij algoritama i algoritama za Kvantno masinsko ucenje.
Samo treba pripaziti kod pominjanja ovih algoritama, jer neki od njih koriste neke koncepte koji su samo
teorijske prirode ili su tesko kreirani u realnom svetu (npr. QRAM).

\section{Kvantna teorija kompleksnosti}
U klasicnoj teoriji kompleksnoti klasifikuju se algoritamski problemi po njihovoj tezini resavanja.
Problemi se klasifikuju u \textbf{klase kompleksnosti}, oni se mogu posmatrati kao kolekcija algoritamskikh problema koji dele neku zajednicku osobinu
vezanu za potrebna komputaciona sredstva potrebna da bi se oni resili (uglavnom vreme i prostor). \cite{nielsen_chuang_10th}\\

\subsection{Primeri klasa kompleksnosti}
Medju kojima su najpoznatiji i najvazniji \textbf{P} i \textbf{NP}. Zbog prirode kvantnih racunara objasnicu neke druge klase kompleksnosti.

Klasa kompleksnosti \textbf{PSPASE} je klasa problema koja se mogu resiti u polinimijalnom prostoru, ali sa neogranicenim vremenom izvrsavanja.


Klasa kompleksnosti \textbf{PP} (Probabilistic Polynomial-Time) je klasa problema za koje postoji nasomicni algoritam u polinomijalnom vremenu koji vraca
tacno resenje sa verovatnocom vecom od $\frac{1}{2}$.

Klasa kompleksnosti \textbf{BPP} (Bounded-Error Probabilistic Polynomial-Time) je klasa problema za koje postoji nasomicni algoritam u polinomijalnom vremenu koji vraca
tacno resenje sa verovatnocom vecom od $\frac{2}{3}$. \cite{aaronson2013quantum}
\subsection{BQP}
Klasa komepleksnoti \textbf{BQP} (Bounded-Error Quantum Polynomial-Time) je klasa problema koji se mogu efikasno resiti na kvantnom racunaru, ako se dopusti ogranicena verovatnoca greske \cite{nielsen_chuang_10th}.
Formalija definicija bi bila:\cite{aaronson2013quantum}
\begin{definition}
    \textbf{BQP} je klasa jezika $L \subseteq \{0,1\}^{*}$ za koje postoji uniformni skup kvantnih kola polinomijalne velicine ($C_n$)
    tako da za svako $x \in \{0,1\}^{n}$:
    \begin{itemize}
        \item ako $x \in L$ onda $C_n$ prihvata ulaz $\Ket{x}\Ket{0...0}$ sa verovatnocom vecom od $\frac{2}{3}$.
        \item ako $x \notin L$ onda $C_n$ prihvata ulaz $\Ket{x}\Ket{0...0}$ sa verovatnocom ne vecom od $\frac{1}{3}$.
    \end{itemize} 
\end{definition}
Ovako definisano moze se primetitni da problemi iz \textbf{BQP} su dosta blizi problemima iz \textbf{BPP} nego iz \textbf{P}.
\subsubsection*{Odnos sa klasnicnim klasam kompleksnosti}
Prva stvar koja vazi jeste da $\textbf{BPP} \subseteq \textbf{BQP}$, odnosno da sve sto mozes da uradis sa klasicnom probalisticnim racunarom moze da 
se uradi i na kvantnom racunaru.

Kada se trazi gornja granica kvantnih problema prvo se dolazi do $\textbf{BQP} \subseteq \textbf{EXP}$, ovo znaci da kvantni racunari mogu da dovedu
do najvise \textit{eksponencijalnog ubzanja} u odnosu na klasicni racunar \cite{aaronson2013quantum}.
Bolja gornja granica za $\textbf{BQP}$ jeste $\textbf{BQP} \subseteq \textbf{PP}$. Ovo su dokazali Adleman, DeMarrais i Huang u \cite{adleman1997quantum}
\subsection{Primeri kvantnih algoritama}
\subsubsection{Šorov algoritam}
Posto je trenetno najoptimalniji algoritam na klasičnom računaru (general number filed sieve) se izvršava u sub-ekponencijalnom vremenu.
Šorov algoritam rešava problem nalaženja periode, koji može da se iskoristi za problem faktorizacije. \cite{QiskiShoreAlgo} \\
\textbf{Problem}: ako imamo periodicnu funkciju
\[
    f(x) = a^x \bmod N
\]
gde su $a$ i $N$ prirodni brojevi,$a < N$ i nemaju zajednickog faktora. Period $r$ je najmanji prirodan broj tako da:
\[
    a^r \bmod N = 1
\]
Sorov algoritam koristi estimaciju kvantne faze na unitarnom operatoru:
\[
    U\Ket{y} = \Ket{ay \bmod N}
\]
Tako da bi eigen vrednost za U bila jednaka superpoziciji stajna $\Ket{u_0}$
\[
    \Ket{u_0}=\frac{1}{\sqrt{r}}\sum_{k=0}^{r-1}\Ket{a^k \bmod N}
\]
\[
    U\Ket{u_0} = \Ket{u_0}
\]
Ovde eigen stanje ima eigen vredost 1, sto nam nista ne znaci. Zanimljivije je kada se gledaju eigen stanja
gde su faze drugacije za svako bazno stanje. Posebno se posmatraju slucajevi gde faza k-tog stanja je proporciona $k$:
\[
    \Ket{u_1}=\frac{1}{\sqrt{r}}\sum_{k=0}^{r-1}e^{-\frac{2\pi i k}{r}}\Ket{a^k \bmod N}
\]
\[
    U\Ket{u_1} = e^\frac{2\pi i}{r}\Ket{u_1}
\]
U ovom slucaju eigen vrednost sadrzi $r$. Ovde $r$ kao faktor normalizacije izmedju $r$ baznih stanja.
Sada, mozemo ova stanja da pomnozimo sa $s \in N_0$ koji ce uticati na krajnju eigne vrednost.
\[
    \Ket{u_s}=\frac{1}{\sqrt{r}}\sum_{k=0}^{r-1}e^{-\frac{2\pi isk}{r}}\Ket{a^k \bmod N}
\]
\[
    U\Ket{u_s} = e^\frac{2\pi is}{r}\Ket{u_s}
\]
Sada imamo jedinstvene eigne stanja za svako $s$ ($0 \leq s \leq r-1$). Ako saberemo sva dobijena eigen stanja, 
razlike u fazama se ponistavaju medjusobno tako da se dobije stanje $\Ket{1}$
\[
    \frac{1}{\sqrt{r}}\sum_{s=0}^{r-1}\Ket{u_s} = \Ket{1}
\]
Posto je bazno stanje $\Ket{1}$ superpozicija za data eigen stanja, sto znaci da mozemo da uradimo \textit{kvantnu esitmaciju faze} nad $U$
koristeci se stanjem $\Ket{1}$, u tom slucaju cemo da izmerimo fazu $\Phi$
\[
    \Phi = \frac{s}{r}
\]
\subsubsection{SWAP Test}
Ova rutina je jednostavan kvantni algoritam koji izrazava skalarni prodakt za dva ulazna kvantna stanja $\Ket{a}$ i $\Ket{b}$. \cite{fastovets2019machine}
\begin{center}
\begin{quantikz}[row sep={10mm,between origins}]
    \lstick{$\Ket{0}$} & \gate{H} & \ctrl{} & \gate{H} & \meter{} \\
    \lstick{$\Ket{a}$} & \qw & \swap{-1} & \qw & \qw \\
    \lstick{$\Ket{b}$} & \qw & \swap{-1} & \qw & \qw
\end{quantikz}
\end{center}

Verovatnoca da se pri merenju kontrolnog kubita dobije stanje $\Ket{0}$ je definisano kao:
\[
  P(\Ket{0}) = \frac{1}{2} + \frac{1}{2}F(\Ket{a},\Ket{b})  
\]
gde je $F(\Ket{a},\Ket{b}) = |\Braket{a|b}|^2$ - validnost izmedju dva kvantan stanja.
Verovatnoca $P(\Ket{0}) = 0.5$ znaci da su kvantan stanja  $\Ket{a}$ i $\Ket{b}$ su medjusobno ortogonalna, 
a verovatnoca $P(\Ket{0}) = 1$ znaci da su kvantan stanja identicna. Ova rutina treba da se ponavalja vise puta
da bi se dobila dobra estimacija vrednosti validnosti. \\
SWAP test se moze koristiti za izracunavanje Euklidske distanfce izmedju kvantnih stanja u visedimenzionom prostoru kao i u velikom 
broju kvatninh algoritama.

\section{Kvantno masinsko ucenje}
Kvantno masinsko ucenje je spoj kvantnih racunara i masinskog ucenja. U programima Kvantnog masinskog ucenja koriste se kvantni algoritmi (npr. qBLAS algoritmi)
kao deo metoda optimizacija slicne klasicnim metodama masinskog ucenja. 

Prema vrsti podataka koji se obradjuju oblast mozemo da dalimo na dve podoblasti
\begin{enumerate}
    \item Obrada klasicnih podataka na kvantnim masinama (\textbf{Masinsko ucenje dopunjeno kvantnim racunarima} eng. Quantum-enhanced machine learning)
    \item Obrada kvantnim podataka na kvantnim masinama
\end{enumerate}
Problem kod obrade klasicnih podataka na kvantnim masinama jeste ucitavanje podataka u sistem, kao i citanje rezultata. Ovo dovodi da algoritnim sa teorijskim eksponencijalnim
ubrzanjem, u realnom svetu budu dosta sporiji i fizcki zahtevniji (velicina kvatnog kola zna da poraste i na skalu oko $10^{25}$ za jednostavnu implementaciju HHL algoritma). \cite{Quantum_machine_learning}

\subsection{Quantum support vector machine}
Jedan od nejednostavnijih primera metoda Kvantnog masinskog ucenja jeste \textbf{Quantum support vector machine} (QSVM). Klasican SVM je metoda koja pronalazi optimalnu podelu hiper-ravni
izmedju dva razlicita skupa podataka, tako da sa velikom verovatnocom svi podaci iz jednog skupa podataka ce se naci na jednoj polovini hiper-ravni. \cite{Quantum_machine_learning}
\subsubsection{Klasican algoritam}
Ova metoda odredjuje klase koristeci linarnu funkciju $w^{T}x + b$. SVD predvidja prvu klasu ako je izlaz funkcije je pozitivan, a predvidja drugu klasu je izlaz negativan.
Posto kod vecina slucajeva odvojenost izmedju dve klase podataka nije linearzibilno odvojivo, sa SVM metodom koristi se i \textbf{Kernel metoda}. \\
Pronalazenje optimalne hiper-ravni se sastoji od minimizacije $|w|^{2}/2$ u nejednacini $y_j(w*x_j+b) \geq 1$ za svako j. 
Ovo minimizicija se moze uraditi, ako uvedemo Karush-Kuhn-Tucker mnozioca $\overrightarrow{\alpha} = (\alpha_1,...,\alpha_M)$ i maksimizujemo ih nad Lagranzovoj funkcijom:
\[
    L(\overrightarrow{\alpha}) = \sum_{j=1}^{M}{y_j\alpha_j} - \frac{1}{2}\sum_{j,k=1}^{M}{\alpha_j\alpha_kx_jx_k}
\]
Sa sledecim ogranicenjima $\sum_{j=1}^{M}{\alpha_j=0}$ i $\forall j \leq M $ $y_j\alpha_j \geq 0$. Tako da, parametre za hiper-ravan se izvode kao:
$w = \sum_{j=1}^{M}\alpha_jx_j$ i $b = y_j - wx_j$ (za one $j$ gde vazi da $\alpha_j \neq 0$). Mali broj $\alpha_j$ je razlicitno od nule, takve promenljive se odnose na vectore $x_j$ koji leze na ravni,
ti vektori se zovu \textbf{Support vektori} \cite{rebentrost2014quantum}

Kernel metoda transformise podatke u prostor gde su dve klase linearno odvojive. Metoda se oslanja na to da se linearna funkcija
moze zapisati iskljucivo kao dot prodakt izmedju primera.
\[
    w^{T}x + b = b + \sum_{i=1}^m \alpha_ix^Tx_i
\]
Gde je $x_i$ trening primer a $\alpha$ je vektor koeficijenata. Ovako zapisivanje funkcije nam dozvoljva da zamenim $x$ sa izlazom funkcije $\phi(x)$, a dot prodakt sa funkcijom $k(x,x_i) = \phi(x)*\phi(x_i)$.
Funkcija $k$ se zove \textbf{kernel}, dok funkcija $\psi$ je funkcija koja preslikava podatke iz jednog prostora u drugi. Operator $\langle * \rangle$ predstavlja unutrasnji prodakt ekvivalentno $\phi(x)^T\phi(x_i)$. \cite{goodfellow2016deep}

Kada zamenimo dot prodakt sa kernelom, funkciju predikcije mozemo da zapisemo kao
\[
    f(x) = b + \sum_i \alpha_{i}k(x,x_i)
\]
Jedan od velikih mana kernel metode jeste cena evaluacije izlaza kernel funkcije je linarna u odnosu na broj trening primera, jer $i$-ti bi oznacavao clana $\alpha_ik(x,x_i)$ kernel funkcije. \cite{goodfellow2016deep} \\
Slozenost SVM je $O(log(1/\epsilon)M^2(N+M))$, gde je $\epsilon$ preciznost resenja, $N$ je broj dimenzija prostora nad kojem radimo ,a $M$ je broj trening primera. \\
Takodje krajnje resenje se je binarni klasifikator za neki vektor $x$:
\[
    y(x) = sign(\sum_{j=1}^{M}\alpha_jkk(x,x_j) + b)
\]

\subsubsection{Kvantni algoritam}
Pretpostavimo da imamo metodu za treniranje(eng. Oracle) koja vraca norme $|x_j|$, labele $y_j$ 
i kvanten vektore $\Ket{x_j} = \frac{1}{|x_j|}\sum_{k=1}^{N}(x_j)_k\Ket{k}$. \\
Bitno nam je za algoritam da ova metoda vraca podatke pod donjom granicom, da bi se kompleksost jezgra algoritma mogla iskazati.
Koristeci evaluaciju inner prodakt priprema se kernel matrica, moze se dobiti SVD algoritam kompleksnoscu $O(log(1/\epsilon)M^3 + M^{2}\log(N/\epsilon))$
Kernel matrica je od velike vaznosti za reformulaciju algoritma kao funkciju kvadratnog troska. 
Uvodimo simplifikaciju za nejednakosti, tako sto uvocimo promenljivu $e_j$ i koristimo osoboinu labela da $y_j^2=1$
\[
    y_{j}(w \cdot x_j + b) \geq 1 \to (w \cdot x_j + b) = y_{j} - y_{j}e_{j}
\]
Porod ove jednacine imamo i implicitan uslov Lagranzove funkcije da sadrzi taksanu (eng. penalty) promenljvi
$\gamma/2\sum_{j=1}^{M}e_j^2$ gde definisana $\gamma$ za relativne tezinu greske treniranja.
Ako uzmemo parcijalno derivat od Lagranzove funkcije i eliminisemo promenljivu $u$ i $e_j$ dovodi do aprokcismaciju funkcije kvadratnog troska problema:
\[
    F\begin{bmatrix}
        b \\
        \overrightarrow{\alpha}
    \end{bmatrix}
    \equiv \begin{bmatrix}
        0 & \overrightarrow{1}^T \\
        \overrightarrow{1} & K+\gamma^{-1} \mathbb{1}
    \end{bmatrix}
    \begin{bmatrix}
        b \\
        \overrightarrow{\alpha}
    \end{bmatrix} =
    \begin{bmatrix}
        0 \\
        \overrightarrow{y}
    \end{bmatrix}
\]
Ovde $K_{ij} = x_i^T \cdot x_j$ je simetricna kernel matrica, $y = (y_1,...,y_m)$ kao i
$\overrightarrow{1} = (1,...,1)$. Matrica $F$ je dimenzija $(M+1)\times(M+1)$. Dodatna dimenzija (red i kolona) se sastoji od jedinica,
zbog offset-a $b$. Promenljiva $\alpha_j$ ima ulogu odredjivanje distance od optimalnog resenja. Tako da na resenje, odnonsno
pronalazenje promenljivih za SVM je oblika:
\[
    \begin{bmatrix}
        b \\
        \overrightarrow{\alpha}
    \end{bmatrix} =
    F^{-1} \begin{bmatrix}
        0 \\
        \overrightarrow{y}
    \end{bmatrix}
\]
U klasicnom algoritmu kopleksnost SVM sa funkcijom kvadratnog troska je $O(M^3)$ 

U kvantnom algoritmu, zadatak je generisanje stanja $\Ket{b,\overrightarrow{\alpha}}$ koja opisuju hiper-ravan i onda klasifikuju stanja
$\Ket{x}$. U algoritmu, resavamo normalizovanu jednacinu $\hat{F}\Ket{b,\overrightarrow{\alpha}} = \Ket{y}$, gde je $\hat{F} = F/trF$ sa 
ogranicenjem $\Vert F \Vert \leq 1$. Klasa ce biti odredjenja kao verovatnoca uspeha pri swap testu izmedju $\Ket{b,\overrightarrow{\alpha}}$ i
$\Ket{x}$. Za efikasnost merenje algoritma, posebno izracunavanja interzne matrice, matrica $\hat{F}$ mora da se razdvoji na jednostavne elemente.
Tako da matrica $\hat{F}$ moze da se razdvoji na sledece elemente $\hat{F} = (J+K+\gamma^{-1}\mathbb{1})/trF$. 
Gde je matrica
\[
    J = \begin{bmatrix}
        0 & \overrightarrow{1}^T \\
        \overrightarrow{1} & 0 \\
    \end{bmatrix}
\]
Takodje, za estimaciju faze pravimo formulaciju Lijevog prodakta \\
$e^{-i\hat{F}\varDelta{t}} = e^{-i\gamma^{-1}\mathbb{1}\varDelta{t}/trF}e^{-iJ\varDelta{t}/trF}e^{-iK\varDelta{t}/trF} + O(\varDelta{t})$

Za njega vazi da ima dve eigen vrednosti oblika $\lambda_{\pm} = \pm \sqrt{M}$,a, istovetno, 
eigen stanja su oblika $\Ket{\lambda_{\pm}} = \frac{1}{\sqrt{2}}(\Ket{0} \pm \frac{1}{\sqrt{M}}\sum_{k=1}^M\Ket{k})$.
Za matricu $\gamma^{-1}\mathbb{1}$ dve eigne vrednosti su $v_1 = 0$ i $v_2 = \gamma^{-1}M$.
Sada mozemo da aproksimiramo fazu za $e^{-i\hat{F}\varDelta{t}}$.

Prvi korak,Stanje $\Ket{y}$ moze da se transformise u eigen state $\Ket{u_j}$ matrice $\hat{F}$, koja ima eigen vrednost $\lambda_j$.
Ono je obilka $\Ket{y} = \sum_{j=1}^{M+1}\braket{u_j|y}\Ket{u_j}$. Ako inicijalizujemo aproksimaciju eigen vrednosti na $\Ket{0}$, i primenimo 
estimaciju faze nad stanjem dobicemo stanje blize pravoj eigen vrednosti:
\[
  \Ket{y}\Ket{0} \to \sum_{j=1}^{M+1}\braket{u_j|y}\Ket{u_k}\Ket{\lambda_j} \to \sum_{j=1}^{M+1}\frac{\braket{u_j|y}}{\lambda_j}\Ket{u_j}
\]
Drugi korak je da invertujemo dobijeno stanje eigen vrednosti, pozivajuci rotaciju stanja. Na kraju dobijamo novo stanje sa trazenim parametrima SVM ($C = b^2 + \sum_{k=1}^M{\alpha_k}^2$)
\[
    \Ket{b,\overrightarrow{\alpha}} = \frac{1}{\sqrt{C}}(b\Ket{0}+ \sum_{k=1}^M{\alpha_k \Ket{k}})
\]

\paragraph*{Klasifikacije}
Sada imamo trenirani model kvantnog SVM-a i zelimo da klasifikujemo stanje $\Ket{x}$. Od stanja $\Ket{b,\overrightarrow{\alpha}}$, koriscenjem metode za treniranje, konstruisemo stanje:
\[
    \Ket{\tilde{u}} = \frac{1}{\sqrt{N_u}}(b\Ket{0}\Ket{0} + \sum_{k=1}^M{\alpha_k |x_k| \Ket{k}\Ket{x_k}})
\]
Gde nam je $N_u=b^2+\sum_{k=1}^M{\alpha_k^2 |x_k|^2}$. Pored ovoga konstruisemo i ulazno stanje $\Ket{\tilde{x}}$:
\[
    \Ket{\tilde{x}} = \frac{1}{\sqrt{N_x}}(\Ket{0}\Ket{0} + \sum_{k=1}^M{|x|\Ket{k}\Ket{x}})
\]
Gde nam je $N_x=M|x|^2 + 1$. Konstruisemo dva nova stanja $\Ket{\psi}$ i $\Ket{\phi}$; $\Ket{\psi}=\frac{1}{\sqrt{2}}(\Ket{0}\Ket{\tilde{u}}+\Ket{1}\Ket{\tilde{x}})$ i
$\Ket{\phi} = \frac{1}{\sqrt{2}}(\Ket{0}-\Ket{1})$. Merenjem swap testa, verovatnoca dobivanja pozitivne vrendosti je $P=|\braket{\psi|\phi}|^2=\frac{1}{2}(1-\braket{\tilde{u}|\tilde{x}})$. 
Ovde unutrasnji produkt, odnosno 
$\braket{\tilde{u}|\tilde{x}} = \frac{1}{\sqrt{N_xN_u}}(b+\sum_{k=1}^M{\alpha_k |x_k||x|\braket{x_k|x}})$, koji se obicno izracunava u $O(1)$ na kvatnom racunaru.
Ako hocemo preciznost $\epsilon$, treba da iteriramo kroz algoritam merenja $O(P(1-P)/\epsilon^2)$ puta. \cite{rebentrost2014quantum}
\subsection{Quantum principal component analysis}
Ova je metoda koja se koristni za smanjivanje dimenzija vektora podataka gde nam je bitno da sacuvamo sto vise informacije o podatku -
labava kompresija (eng. lossy compression). 
\subsubsection{Klasicni algoritam}
Neka za svaku tacku $x^{(i)} \in \mathbb{C}^n$ zelimo da transformisemo u tacku $c^{(i)} \in \mathbb{C}^l$ gde je $l < n$.
Zelimo da nadjemo funkciju enkodovanja koja za ulaz $x$ vraca $c$, odnosno, $f(x)=c$. Takodje zelimo da nadjemo funkciju dekodovanja $g(f(x)) \approx x$.

Zbog jednostavnosti, uzecemo funkciju mnozenja matrica kao funkciju dekodavanja. Neka je $g(c)=Dc$, gde je
$D \in \mathbb{C}^{n \times l}$ matrica definisana za dekodovanje. Takodje zbog optimalno izracunavanja funkcije enkodovanja, 
PCA uvodi ogranicenje da su kolone medjusobno ortogonalne. Jos jedno ogranicenje koje moze da se uvede u algoritam,
i koji ce dovesti do jedinstvenog resenja, jeste da su sve kolone matrice $D$ u unitarnoj normi.
Jedan od nacina na koji hocemo da nadjemo optimalnu projekciju $c$ za ulaz $x$ jeste da nadjemo najmanju L2 distancu izmedju
ulaza $x$ i dekodovane vrednosti $g(c)$
\[
  c^{*} = \underset{c}{\mathrm{argmin}} \Vert x-g(c) \Vert_2^2
\]
I ova za pronalazenje minimalne distance ce dovesti do optimalnog resenje $c=D^{T}x$\cite{goodfellow2016deep}. Tako da funkcija enkodovanja je oblika:
\[
f(x) = D^Tx  
\]
Takodje, mozemo da uvedemo novu funkciju rekonstrukcije ulaza $x$
\[
  r(x) = g(f(x)) = DD^Tx  
\]
Sada treba da se nadje optimalna matrica $D$. Ovo ce se resiti na isti nacin kao i pronalazenje optimalnog $c$ za ulaz $x$, odnostno kao pronalazenje minimalne L2 distance za ulazne vektore
njihove rekonstrukcije.
\[
    D^{*} =\underset{D}{\mathrm{argmin}} \sqrt{\sum_{i,j}(x^i_j - r(x^i)_j)^2} \text{ gde vazi } D^TD = I_l
\]
Posle procesa izvodjenja \cite{goodfellow2016deep}, jendacina za optimalnu matricu $D$ je oblika:
\[
    D^{*} =\underset{D}{\mathrm{argmin}}  Tr(D^TX^TXD) \text{ gde vazi } D^TD = I_l
\] 
Gde nam je $X \in C^{m \times n}$ matrica gde su redovi ulazni vektori $x$. Ova jednacna se moze resiti koristeci eigen dekompoziciju. Gde bi 
se pronasli eigen vektor za $X^TX$ za najvecu eigne vrednost. 
\subsubsection{Kvantni algoritam}
U kvantnom algoritmu bitno nam je da nedjenmo eigen vektore i eigen vrednosti za ulaz. Ovo se dosta olanja na drugi deo metode koji je opisan u Support vector machine sekciju.
Ako izaberemo random vektor $v_j$ iz skupa ulaznih vektora,kreiramo kvanto stanje $\Ket{v_j}$; tada mozemo da kreiramo \textit{density} matricu $\rho = (1/N)\sum_j\Ket{v}\Bra{v}$ gde je $N$ velinica skupa vektora. \cite{Quantum_machine_learning}
Slicno \textbf{qSVM} nad \textit{density} matricom $\rho$ mozemo da apliciramo algoritam esitacije faze stanja. Odnosnto, da primenimo $e^{-i \rho t}$, $t$ puta nad inicijalnim stanjem:
\[
    \Ket{v_j}\Ket{0} \to \sum_i \psi_i\Ket{\chi_i}\Ket{\widetilde{r_i}}
\] 
Gde je $\Ket{\chi_i}$ eigen vektor od matrice $\rho$, $\widetilde{r_i}$ je esimacija eigen vrednosti, a $\psi_i = \Braket{\chi_i|v_j+}$.
I primenom SWAP testa na dobijenim stanjem dobijamo stanje:
\[
    \sum_i r_i \Ket{\chi_i}\Bra{\chi_i}\otimes \Ket{\widetilde{r_i}}\Bra{\widetilde{r_i}}
\]
Merenjem ovog stanja mi dobojamo eigen vrednost i eigen vektor za \textit{density} matricu $\rho$. Ako uradimo 
ovaj proces nad vecem brojem kopija matrice $\rho$, dobicemo preciznije estimacije eigen vrednosti i eigen vektora. \\
Sada kada imamo eigen vrednost i eigen vektor mozemo da rekonstruisemo matricu za enkodovanje $D$.
Vremenska slozenost ovog algoritma je $O(\log d)$. \cite{Lloyd_2014}

\subsection{Kvantna neuralna mreza}
Neuralne mreze su osnova polja koji se naziva \textbf{Duboko ucenje} i zato postoji veliku paznja za razvoj istog.
U papiru \cite{Classification_wit_QNN}, autori su predstavili osnove algoritama za Kvantnu neuralnu mrezu (QNN).
Da li su neke primere, neke prednosti i neke nedostatke kvantnog pristupa neuralnim mrezema\\

Neka unani skup stringova $\phi$ oblika $z=z_1 z_2 \dots z_n$ gde svako $z_i$ je bit cija vrednost moze da bude $+1$ ili $-1$,
kao i binarnu oznake $l(z)$ koje moze da bude $+1$ ili $-1$. Zbog jednostavnosti neka se u nasem setu nalazi sve permutacije 
ovako opisanog stringa, to jest neka $|\phi|=2^n$.
Predstavicemo kvantni proces koji radi na $n+1$ kubita (poslednji kubit sluzi kao izlaz procesa). Kvantni proces se sastoji od 
unitarnih transformacija ulaznih stanja: ${U_a(\theta)}$.
Svaka transformacija radi nad podskupu ulaznih kubita i zavisi od promenljive $\theta$.
Sada izabracemo podskup od $L$ transformacija:
\[
  \mathbf{U}(\overrightarrow{\theta}) = U_{L}(\theta_{L}) U_{L-1}(\theta_{L-1}) \dots U_{1}(\theta_{1}) 
\]
koja zavise od $L$ parametara $\overrightarrow{\theta}=\theta_{L} \theta_{L-1} \dots \theta_{1}$.
Za svaki string $z$ kreiracemo pocetno stanje:
\[
    \Ket{z,1} = \Ket{z_1, z_2, \dots z_n, 1}
\]
Primenjivanje unitarne transformacije vraca stanje: $U(\overrightarrow{\theta})\Ket{z,1}$
Na izlazu meri se dodati kubit sa Puali-jevim operatorom $\sigma_y$, koji se kasnije naziva i $Y_{n+1}$.
Tako da na kraju imamo izlaz $+1$ ili $-1$. Cilj je isti kao i kod klasicnih neuralnih mreze da "naucimo" proces da vraca
tacne vrednosti za dati ulazni string. Posto merenje izlaznog kubita nije sigurno, odnosno merenje kubita dobijamo tacnu vrednost 
sa nekom verovatnocom uvodimo transformaciju:
\[
    \bra{z,1}U^T(\overrightarrow{\theta})Y_{n+1}U(\overrightarrow{\theta})\Ket{z,1}
\]
koji predstavlja prosecnu vrednost merenja, ako $Y_{n+1}$ merimo na vise kopija originalno izlaza. 

Ovde, kao i u klasicnoj neuralnoj mrezi, cilj nam je da nadjemo parametar $\overrightarrow{\theta}$ koja vraca tacnu vrednost sa velikom preciznoscu.
Slicno kao i prethodnoj postavci imamo: $L$ unitarnih promenljivi sa korespodentnim promenljivama $\overrightarrow{\theta}$, kao i ulazni string $z$; 
tada mozemo da predstavimo funkciju troska:
\[
    \mathit{loss}(\overrightarrow{\theta},z) = 1 - l(z)\bra{z,1}U^T(\overrightarrow{\theta})Y_{n+1}U(\overrightarrow{\theta})\Ket{z,1}
\]
Mozemo primetiti da ova funkcija troska je linearna i da je minimum u $0$, jer je vracema vrednost izmedju $-1$ i $+1$.
Ako pretpostavimo da kvantna neuralna mreza radi savrseno, tako da za svaki ulazni string $z$, merenje uvek vraca tacnu oznaku.
To onda znaci da optamalna promenljiva $\overrightarrow{\theta}$ postoji i da je minimum za funkciju trosa u $0$ za sve ulaze $z$.

Neka imamo skup stringova \(S\) za treniranje, sa njihovim oznakama. Postoji kvantni proces koji ima mogucnost da
prikaze trazene labele i zavisi od parametara $\overrightarrow{\theta}$. Opisacemo proces kako da dodjemo do optimalnih parametara $\overrightarrow{\theta}$.
Neka pocnemo sa random promenljivom $\overrightarrow{\theta}$ (ili ako imamo neku pretpostavku vrednosti parametara). Izaberimo neki string $z^1$ iz skupa za traniranje.
Primenjujemo kvanti proces nad izabranim stringom:
\[
    U(\overrightarrow{\theta})\Ket{z,1}
\] 
i merimo $Y_{n+1}$ na zadnjem kubitu. Nakon nekoliko merenja mozemo da imamo dobru aproksimaciju ocekivane vrednosti od $Y_{n+1}$
i tada izracunavamo $\mathit{loss}(\overrightarrow{\theta},z^1)$. Nakon toga, zelimo da promenimo parametar$\overrightarrow{\theta}$ tako da smanjimo
funkciju troska za string $z^1$. Postoje dva nacina da se uradi trazeno: (1) da uradimo pomeraj po nekom uzimanju uzorka u $[\overrightarrow{\theta}-\epsilon,\overrightarrow{\theta}+\epsilon]$ intervalu.
(2) da izracunamo derivat funkcije troska po $\overrightarrow{\theta}$ i da se malo pomerimo ka pravcu koji smanjuje funkciju.
Ovo nam daje novi parametar $\overrightarrow{\theta^1}$. Sada biramo ponovo iz skupa neki string $z^2$ i ponovimo prethodni proces ali sa parametrom $\overrightarrow{\theta^1}$.
Ovako dobijamo novi parametar $\overrightarrow{\theta^2}$ koji ima manju funkciju troska za string $z^2$ nego parametar $\overrightarrow{\theta^1}$.
Ovako prolazimo kroz proces sve dok ne prodjemo kroz ceo skup $S$. Kao rezultat ovoga generisali smo sekvencu parametara
$\overrightarrow{\theta^1}, \overrightarrow{\theta^2}, \dots \overrightarrow{\theta^S}$. Ako nam je "ucenje" parametara uspesno onda bi smo dobila
da operator $U(\overrightarrow{\theta^S})$, kada se primeni na stanju $\Ket{z,1}$, vratice stanje koje kada se izmeri na izlazu vraca tacnu oznaku $l(z)$.
Ako je $z$ iz skupa za traniranje, reci cemo da je model fitovao podatke za treniranje. Ako je $z$ izvan skupa za treniranje, mozemo raci da je model naucio da generalizuje i za nevidjenje podatke.

Ovaj proces koji je opisan, primeti ce te, u klasicnom masinskom ucenje zove se "Stohasticko uvenje".
U tradicijonalnom masinskom ucenju sa neuronskim mrezama, parametri se prikazuju kao promenljive unutar matrice, koja 
je linarna u odnosu na unutrasnje vektore. Nad Komponentama tih vektora vrsi se nelinearne transformacije, pre nego sto se mnoze sa
ostalim parametrima. Uvedjenje dobre ne linearnosti je jedan od glavnih delova uspesne implementiacije modela u klasicnom masinskom ucenju.
Ovu osobinu klasnih neuralnim mreza tesko je prebaciti u kvantni sistem, jer je kvantna mehanika, osnova celog koncepta kvantnog racunarstva, samo po sebi linearna.
U metodi koja je opisana, svaka unitarna opearcija se izvrsava nad izlazom prethodne operacije, pri cemu se izmedju operacija ne izvrsava nikakva nelinearna transformacije.
Neka name je svaka unitarne transformacija oblika $e^{i\theta\Sigma}$, gde je $\Sigma$ produkt tenzora koji se sastoji iz skupa Paulijevih operatora, i rade nad nekolicinom kubita.
Derivat operatora po $\overrightarrow{\theta}$ je ogranica po $L$, to jest po broju parametara.
Ovo je znacajno, jer znaci da gradijent ne moze da ode u beskonacno i tako izbegavamo veliko problem koji se moze desiti klasicnim neuralnim mrezama.

\subsubsection{Reprezentacija modela}
Neka imamo $2^n$, $n$-bitnih stringova i vezano za njih postoje $2^{(2^n)}$ funkcija oznaka $l(z)$.
Ako nam je data odredjena funkcija oznaka onda mozemo da definisemo operator nad komputacionim osnovama kao:
\[
    U_l\Ket{z,z_{n+1}} = e^{i\frac{\pi}{4}l(z)X_{n+1}}\Ket{z,z_{n+1}}
\]
Ovaj operator rotira ulazni kubit oko $x$-ose za $\frac{\pi}{4}$ puta oznaka za string $z$.
Tako da iz toga imamo:
\[
    U_l^TY_{n+1}U_l = cos(\frac{\pi}{4}l(Z))Y_{n+1} + sin(\frac{\pi}{4}l(Z))Z_{n+1}
\]
gde u formuli $l(Z)$ je interpretirana kao operator dijagonalan u odnosu na komputaciona osnovna stanja.
Takodje, posto funkcija oznaka l(z) moze da vrati ili $+1$ ili $-1$ iz toga imamo $\bra{z,1}U_l^TY_{n+1}U_l\ket{z,1} = l(z)$.
Ovo nam pokazuje da bar na nekom abstraktnom nivou imamo mogucnost da predstavimo bilo koju funkciju oznake kao kvantno kolo.

Objasnjenje kako da se napise operator $U_l$ kao produkt dve kubit unitarne transformacije. Zbog ovoga treba da se predje na \textit{boolean} promenljive $b_i=\frac{1}{2}(1-z_i)$ i 
neka funkcija oznake $l$ bude oblika $1-2b$ gde je $b \in {0,1}$. Sada mozemo da iskoristimo \textbf{Reed-Muller} iskazivanje bilo koje \textit{boolean} funkcije u obliku bitova $b_1 \dots b_n$:
\[
    b = a_0 \oplus (a_1b_1 \oplus a_2b_2 \oplus \dots a_nb_n) \oplus (a_{12}b_1b_2 \oplus a_{13}b_1b_3 \oplus \dots) \oplus \dots \oplus a_{12 \dots n}b_1b_2 \dots b_n
\]
gde su koeficijenati $a \in {0,1}$. Primecuje se da imamo $2^n$ koeficijenta i posto su oni ili 0 ili 1 da stvarno imao
$2^{(2^n)}$ mogucih \textit{boolean} funkcija. Nasa funkcija $b$ moze biti ekoponencijalno dugacka.
Sada mozemo da zapisemo unitarnu transformaciju koja zavisi od funkcije oznaka kao:
\[
    U_l = e^{i\frac{\pi}{4}X_{n+1}}e^{-1\frac{\pi}{2}BX_{n+1}}
\]
gde je $B$ operator, dijagonalan u odnostu na kompuntacione baze, koji odgovara nama data funklcija $b$. 
Svaka vrednostu u $B$ se mnozi sa $X_{n+1}$ tako da svaka vrednost je komutativna sa ostalim vrednostima. 
Svaka clan, razlicit od nule, u \textbf{Reed-Muller} formuli utice u $U_l$ na kontrolni \textit{bit flip} na izlaznom kubitu.

Ovaj rezultat kvantno reprezentacije ima analog u klasicnoj teoriji \\
reprezentacije \cite{Cybenko1989ApproximationBS}.
Ona pokazuje da bilo koja \textit{boolean} funkcija ozneke moze da se prestavi u neuralnoj mrezi dubine tri, gde srednji slog 
ima velicinu $2^n$. Ovako velika matrica ne bi mogla da se prestavi na klasicnim racunarima, ali na kvantnim racunarima, oni po prirori rade
nad Hilbertovim prostorim sa eksponencijalnim dimenzijama. Ali jos nije dokazano da svaka \textit{boolean} funkcija moze da se prestavi u
kvantno kolo koje nije eksponencionalne dubine. Na tome se trenutno dosta radi u naucnim krugobima.

\paragraph*{Reprezentacija parnosti podskupa}
Neka imamo datu funkciju oznaka koja vraca parnost podskupa bitova datog stringa. Neka je podskup $\mathbb{S}$
i neka je $a_j=1$ ako bit $j$ je u podskupu i $a_j=0$ ako $j$ nije u podskupu. Reed-Muller formula za parnost podskup je:
\[
    P_{\mathbb{S}}(z) = \sum_j \oplus a_jb_j
\]
Ovo nam dozvoljava da napravimo unitarnu transformaciju koja implementira parnost podskupa:
\[
    U_{P_{\mathbb{S}}} = e^{i\frac{\pi}{4}X_{n+1}}e^{-i\frac{\pi}{2}\sum_j a_jB_jX_{n+1}}
\]
Kolo se sastoji od, najvise, $n$ operatora nad dva kubita koji su komutativni medjusobno, gde je pridodati kubit u svim operatorima nad dva kubita.

\subsubsection{Ucenje modela}
U ovoj podsekciji ce se objasniti dve potencijalne metode kako da menja parametar $\overrightarrow{\theta}$ tako da se funkcija troska smanjejue.
Ako su nam dati paramteri $\overrightarrow{\theta}$ i trening primer $z$, prvo procenjujemo vrednost trosak od $\mathit{loss}(\overrightarrow{\theta},z)$.
Da bi smo ovo uradili treba da napravimo vise merenje $Y_{n+1}$ za $U(\overrightarrow{\theta})\Ket{z,1}$.
Da bi smo ovo uspeli sa verovatnocom vecom od $99\%$, procena of funkcije trosa koja je $\delta$ intervaluj od prave vrednosti funkcije troska
treba da napravimo najmanje $2/\delta^2$ merenja($\delta \in (0,1)$).

Nako sto procenimo vrednost funkcije troske zelimo da izracunamo gradijent od funkcije trosa u odnosu na $\overrightarrow{\theta}$.
Jedan od nacina jeste da menjamo jedenu po jednu promenljivu u $\overrightarrow{\theta}$. Nakon svake promene treba da se izracuna
$\mathit{loss}(\overrightarrow{\theta'},z)$, gde $\overrightarrow{\theta'}$ je razlicit od $\overrightarrow{\theta}$ za neku malu vrednost u jednoj promenljivi.
Ako bi se koristio simetrican derivat funkcije troska svaku promenljivu parametra bi mogli da izrazunamo do preciznosti $\eta$ u oko $1/\eta^3$ merenja.
Ovaj proces bi trebao da se ponavlja $L$ puta da bi se dobio puni gradijent.

Alternativna strategija jeste da se menja svaka promenljiva gradijenta, sto se koristi kada su sve unitarne transformacije oblika $e^{i\theta\Sigma}$.
Ako posmatramo derivat za funkciju troska $\mathit{loss}(\overrightarrow{\theta},z)$ za parametar $\theta_k$, koji je vezan za transformaciju $U_k(\theta_k)$
(koja ima i generalni Pauli-jer operator $\Sigma_k$). Sada:
\[
    \frac{d\mathit{loss}(\overrightarrow{\theta},z)}{d\theta_k} = 2Im(\Bra{z,1}U_1^T \dots U_L^TY_{n+1}U_L \dots U_{k+1} \Sigma_k U_k \dots U_1 \Ket{z,1})
\]
Ako primetimo da su $Y_{n+1}$ i $\Sigma_k$ unitarni operatori, tada definisemo unitarni operator:
\[
    \mathcal{U}(\overrightarrow{\theta}) = U_1^T \dots U_L^TY_{n+1}U_L \dots U_{k+1} \Sigma_k U_k \dots U_1
\]
tako da derivat mozemo da zapisemo kao:
\[
    \frac{d\mathit{loss}(\overrightarrow{\theta},z)}{d\theta_k} = 2Im(\Bra{z,1}\mathcal{U}\Ket{z,1})
\]
$\mathcal{U}(\overrightarrow{\theta})$ se mozemo posmatrati kao kvanto kolo koji sadrzi $2L+2$ unitarnih transformacija.
Sada mozemo da primenimo $\mathcal{U}(\overrightarrow{\theta})$ nad stanjem $\Ket{z,1}$. Ako koristimo dodati kubit, mozemo da merimo imaginarni deo
derivata funkcije. Zapocecemo sa stanjem $\Ket{z,1}\frac{1}{\sqrt{2}}(\Ket{0}+\Ket{1})$ i primenicemo $i\mathcal{U}(\overrightarrow{\theta})$ na dodatim kubitom vrednosti $1$.
Ovo kreira:
\[
    \frac{1}{\sqrt{2}}(\Ket{z,1}\Ket{0} + i\mathcal{U}(\overrightarrow{\theta})\Ket{z,1}\Ket{1})
\]
Ako primenimo Hademardovu kapiju na dodatim kubitom dobijamo:
\[
    \frac{1}{2}(\Ket{z,1}+\mathcal{U}(\overrightarrow{\theta})\Ket{z,1}\Ket{0}) + \frac{1}{2}(\Ket{z,1}-i\mathcal{U}(\overrightarrow{\theta})\Ket{z,1}\Ket{1})
\]
Sada kada izmerimo dodati kubit, verovatnoca da se dobije $0$ je:
\[
    \frac{1}{2} - \frac{1}{2}Im(\bra{z,1}\mathcal{U}(\overrightarrow{\theta})\Ket{z,1})
\]
tako da, ponavljanjem ovog merenja mozemo da dobijemo dobru procenu imaginarnog dela stanja iz kojeg mozemo da izvucemo procenu k-te komponente trazenog gradijenta.
Ovaj metod izbegava numbricne nepreciznosti prethodne strategije. Cena ove metode je potreba da dodatim kubitom kao i kvatno kolo $2L+2$ dubine.

Posto samo izracunali gradijent, sada treba metod za izmenu $\overrightarrow{\theta}$. Neka je $\overrightarrow{g}$ gradijent funkcije troska po parametru $\overrightarrow{\theta}$.
Sada menjamo $\overrightarrow{\theta}$ u pravcu $\overrightarrow{g}$. Sa velicinom "koraka" $\gamma$ imamo
\[
    \mathit{loss}(\overrightarrow{\theta}+\gamma \overrightarrow{g}) = \mathit{loss}(\overrightarrow{\theta},z)+ \gamma \overrightarrow{g}^2 + O(\gamma^2)
\]
Posto zelimo da smanjimo trosak na $0$ mozemo da napravimo da:
\[
    \gamma = -\frac{\mathit{loss}(\overrightarrow{\theta},z)}{\overrightarrow{g}^2}
\]
Ovako nesto bi dovelo da trosak bude $0$ za trenutni primer, ali moze da dovede do losih efekata za ostale primera. 
Ovde se u klasicnom masinskom ucenju obicno uvodi promenljiva, stepem ucenja $r \in (0,1]$ i onda imamo sledece:
\[
    \overrightarrow{\theta} \arrowvert \overrightarrow{\theta} - r(\frac{\mathit{loss}(\overrightarrow{\theta},z)}{\overrightarrow{g}^2})\overrightarrow{g}
\]
Deo uspesne implementacije masinskog ucenja je racionalno odaberemo vrednost stepena ucenja.

\paragraph{Ucenje parnosti podskupa}
Za dati podskup $\mathbb{S}$, unitarna transformacija $U_{P_\mathbb{S}}$ moze da prikaze parnost podskupa za sve ulazne stringove.
Na bi se "naucio" skup unitarnih operacija koji zavise od parametara, sa tim da za svaki podskup postoje parametri koji opisuju $U_{P_\mathbb{S}}$.
Najlaksi nacin da se ovo postigne jeste da se koriste $n$ parametara
\[
    U(\overrightarrow{\theta}) = e^{i\frac{\pi}{4}X_{n+1}}e^{-i\sum_j^n \theta_jB_jX_{n+1}}
\]
ovde se vidi da je reprezentacija savrsena kada je $\theta_j = \frac{\pi}{2}$ ako je $j$ u podskupu i$\theta_j = 0$ ako $j$ nije u podskupu.
Posle eksperimenta sa malim brojem kubita gde su uspeli da nauce model, njihov argument je da sa povecanjem velicine sistema postaje nemoguce da se nauci kvantni model.
Da bi to pokazali, izracunali eksplicitnu formulu su ocekivanu vrednost za $Y_{n+1}$
\[
    \bra{z,1}U^T(\overrightarrow{\theta})Y_{n+1}U(\overrightarrow{\theta})\Ket{z,1} = cos(2\sum_j\theta_jb_j)
\]
Sa oznakom $l(z)$ moze se ubaciti u funkciju troska, ali sada moze da se izracuna prosek
troska za sve $2^n$ stringove, jer imamo eksplicitnu formulu za oznake i njihovo ocekivanje.
Postoje vise verzija izracunate funkcije, koja zavise od izlaza $n \bmod 4$ i koliko bitova se nalazi u podskupu $\mathbb{S}$.
Za prikaz uzeli su primer gde je $n$ deljiv sa 4 i skup $\mathbb{S}$ sadrzi svih $n$ bitova.
U tom slucaju prosecan trosak za sve stringove je 
\[
    1 - cos(\theta_1 + \theta_2 + \dots \theta_n)sin(\theta_1)sin(\theta_2) \dots sin(\theta_n)
\]
Iz formule se vidi da je u minimum kada su sve $\theta=\frac{\pi}{2}$. Zamislite kakva bi bila pretraga minimuma (pored ovog primera) funkcije
nad intervalom $[0\;\pi]^n$. Funkcija bi samo prikazivala vrednosti eksponencijalno blizu $1$, sem u eksponencijalno malim intervalima oko optimalnih uglova.
Isto tako gradijent bi bio veoma mali sem oko optimalnih uglova. Zato cak i ako imamo pristup prosecnom trosku, nijedan metod koji se oslanja na gradijentalni pristup
bi mogao biti koriscen za pronalazenje optimalnog ugla za bilo koji primer sa povecim $n$, gradijent bi radio izvan preciznosti masine u tom slucaju.

\subsubsection{Ucenje osobina kvantnih stanja}
Sa kvantnom neuralnom mrezom, ocekuje sa da na ulazu moze da ima bilo koje kvantno stanje (koje nije izvedeno iz nekog klasicnog podatkja)
i da moze da nauci neke njegov osobine i da ih izbaci u obliku nekih oznaka. Ne postoji ni jedna klasicna neuralna mreza koja moze to da uradi, jer klasicni racunari ne mogu da prihvate kvantno stanje kao ulaz.
Osnovna ideja je da se $n$-kubitno stanje $\Ket{\psi}$ ubaci u kvantnu neuralnu mrezu sa dodatim kubitom, koji sluzi za citanje rezultata, koji je postavljen na $1$.
Pa neka nam je data unitarna transformacija $U(\overrightarrow{\theta})$ tako da imamo stanje
\[
    U(\overrightarrow{\theta})\Ket{\psi,1}
\] 
i onda merimo $Y_{n+1}$. Cilj ovoga je da namestimo da izlaz ovog merenje bude ekvivalntan nekim dvema oznaka koje oznacavaju neke osobine kvantog stanja.
To je prikazano u sledecem primeru.
Posmatrajmo Hamiltojev operator $H$ (eng. Hamiltonian), koji je suma lokalnih vredosti sa dodatnom osobino da ima i pozitivne i negetivne eigen vrednosti.
Sa datim kvantnim stanjem $\ket{\psi}$, obelezava se oznakom koja pokazuje da li je ocekivana vrednost Hamiltoijevog operatora pozitivna ili negativna:
\[
    l(\Ket{\psi}) = sign(\bra{\psi}H\Ket{\psi})
\]
Posmatrajmo operator $U_H(\beta) = e^{i\beta HX_{n+1}}$, gde je $\beta$ mala i pozitivna vrednost. Sada
\[
    \Bra{\psi,1}U_H^T(\beta)Y_{n+1}U_H(\beta)\Ket{\psi,1} = \bra{\psi}sin(2\beta H)\Ket{\psi}
\]
tako da ze dovoljno malo $\beta$ ovo je priblizno jednako $2\beta\bra{\psi}H\Ket{\psi}$ i tako imamo znak ocekivane vrednosti za 
predikciju oznake koje je jednata sa tacnom oznakom. Ovako prikazana, ovo je funkcija oznake sa kvantnim kolima koja
ima malu gresku. Mala greska dolazi iz toga sto $\bra{\psi}sin(2\beta H)\Ket{\psi}$ samo priblizno jednako $2\beta\bra{\psi}H\Ket{\psi}$
Ako uzmemo da $\beta$ bude dosta manje od $1/\Vert H \Vert$ (inverz of norme matrice H), mozemo da napravimo da nam greska bude mala.

Posmatrajmo graf gde na svakoj ivici imamo \textit{ZZ} uparivanje sa koeficijenatom ili $+1$ ili $-1$.
Hamiltonijev operator je oblika: $H = \sum_{i,j}J_{ij}Z_iZ_j$ gde prvobitna suma ogranicena na ivice grafa i $J_{ij}$ je ili $+1$ ili $-1$.
Neka postoje $M$ vrednosti u $H$. Prvo, treba da izaberemo $M$ uglova $\theta_{ij}$ i neka je kvantno kolo koje implementira transformaciju oblika:
\[
    U(\overrightarrow{\theta}) = e^{i\sum_{i,j}J_{ij}Z_iZ_jX_{n+1}}
\]
Ako izaberemo $\theta_{ij}=\beta J_{ij}$ imamo operator $U_H(\beta)$ koja osigurava da mozemo da oznacimo trazenu oznaku
ako izaberemo malo $\beta$ \\
Kvantno stanje $\Ket{\psi}$ je u Hilbertovom prostoru sa $2^n$ dimenzija i ne mozemo da ocekujemo da naucimo oznake za svako stanje.
Posmatrani Hamiltojev operator ima bitovnu strukturum, tako da mozemo da se ogranicimo kvantna stanja sa bitnovnim strukturama.
Tako da je predlozeno da se za treniranje koriste stanja samo sa ovom formom i za testiranje isto.
\section{Zakljucak}
U ovom sam vam prikazao kraci uvod u kvantno racunarstvo i kvantno masinsko ucenje. Takođe je dato primer nekoliko metoda kvantnog masinskog ucenja, 
koja poboljšavaju svoje klasične ekvivalente. Glavni izazov trenutno je praktična primena, koja je strogo vezana za razvoj funkcionalnih kvantnih računara.

Ostavljeno je dosta nepokrivenih metoda kvantnog mašinskog učenja, koje čitalac može da istraži.
Savatovao bih praćenje daljeg razvoja ove oblasti, pošto je ona relativno mlada i ima dosta inovacija i poboljšavanja.

\newpage

\printbibliography
\end{document}