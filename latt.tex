\documentclass[12pt, letterpaper, oneside]{article}
\usepackage[utf8]{inputenc}
\usepackage[LGR,T1]{fontenc}
\usepackage{braket}
\usepackage{amsmath}

\title{Kvantno masinsko ucenje}
\author{Milan Bojic}
\date{Jun 2022}

\renewcommand*\contentsname{Sadrzaj}

\begin{document}

\maketitle

\tableofcontents
\newpage
\section{Kvantno racunarstvo}
Pre nego sto se pocne pricati o Kvantnom masinskom ucenju, treba objasniti neki osnovni pojmovi da bi lakse razumeli ostatak rada

\subsection{Osnovni pojmovi}
Potrebni pojmovi su:
\begin{itemize}
    \item Kubit (eng. Qubit)
    \item Kvantna kola (eng. Quantum Gates)
    \item Kvantna uvezanost (eng. Quantum entanglement)
    \item Kvantan memorija, Kvantni registri
\end{itemize}

\subsubsection*{Kubit}
Kubit (eng. Qubit ) je najmanja jedinica informacije u kvantnom računarstvu, slično bit-u u klasičnom računarstvu.
Razlika od bita jeste u tome što kubit pored stanja 1 i 0, može da se nalazi i u superpoziciji između oba.
Oni se mogu predstaviti formulom (koristeci "bra-ket" notaciju):
\[ \braket{\gamma} =  \alpha\braket{0}+\beta\braket{1} \]
Ovde su $\braket{0}$ i $\braket{1}$ zapravo stanja kao i kod klasičnog bita, a $\alpha$ i $\beta$ su kompleksni brojevi koji predstavljaju aplitude zadatih stanja i za njih važi:

\[ |\alpha|^2+|\beta|^2 = 1 \]
Posto stanje kubita ima dva stepena slobode sto dovodi do toga da amplitude se mogu zapisati kao:
\[
    \alpha = \cos{\frac{\Theta}{2}} 
\]
\[
    \beta = e^{i\phi}\sin{\frac{\Theta}{2}}
\]
Takodje mozemo da vidimo da je $|\alpha|^2$ verovatnoca da se kubit nalazi u stanju 0, isto vazi i za $ |\beta|^2$ i 1.
Saznanje o tomo u kom stanju se nalazi kubit ce se dobiti merenjem kubita, tade ce da kubit izadje iz superpozicije i "pasce" u stanje 1 ili stanje 0. U tom slucaku kubit ce imati ponasanje kao i obican bit, ali ovako gubimo predjasnje kvantno stanje kubita.
U fizičkom svetu kubit se moze predstaviti kao polarizovani fotoni, pre cemu se dva stanje se uzimaju kao vertikalna i horizontalna polarizacija.
\subsubsection*{Kvantna kola}
Kvantna kola (eng. Quantum Gates ) su logički predstavljaju matricama i oni rade na određenom broju kubita.Matrice su unitarne sa oblikom $2^n \times 2^n$, gde je $n$ broj qubita na kojim radimo . Neke od poznatih kola su: Hademardovo kolo (stavalja kubit u superpoziciju), bit flip kolo (zamenjuje aplitude na kubitu), ali nas najviše zanima rotaciono kolo:
\[
    R = \begin{bmatrix}
        \cos{\Theta} & -\sin{\Theta} \\
        \sin{\Theta} & \cos{\Theta} 
    \end{bmatrix}
\]
Ovo kolo rotira kubite u prostoru, odnosno menja njihove amplitude za $\Theta$ radiana.

\subsubsection*{Kvantna uvezanost}
Kvantna uvezanost (eng. Quantum entanglement) je fizički pojam gde su dva ,ili više, kubita povezana tako da zajedno prave novo kvantno stanje.
U čistim stanjima oni su matematički zapravo proizvodi tenzora amplituda:
\[
    \braket{\gamma} \otimes \braket{\delta} = \alpha_1\alpha_2\braket{00} + \alpha_1\beta_2\braket{01} + \beta_1\alpha_2\braket{10} + \beta_1\beta_2\braket{11}
\]
I ovako napisano kvantno stanje se moze razdvojiti na dva kubita. Ali postoje i kvanta stanja koja se ne mogu razdvojiti npr.
\[
 \frac{1}{\sqrt{2}}\braket{00} + \frac{1}{\sqrt{2}}\braket{11}
\]
Zanimljiva stvar kod uvazanih kubita jeste u tome što dele informacije. Ako bi jedan kubit iz para odneli u neko veoma daleko mesto (na primer druga galaksija), i tamo bi ga izmerili mi bi smo dobili 0 ili 1, međutim drugi kubit bi takođe upao u određeno stanje i to u istom trenutni kad smo izmerili prvi daleki kubit. Ovo je zapravno gde se nalazi glavan različitost između klasičnog i kvantnog računarstva, ova pojava ne postoji u klasičnom računarstvu i ne može se "lako" simulirati.
\subsubsection*{Kvantni registri}
Kvantni registri se sastoje od kvantnog stanja od $m$ uvezanih kubita i moze da se predstavlja do $2^m$ vrednosti stanja istovremeno.
Kvantan memorija su uredjaji koji cuvaju kvantna stanja fotona, bez da unistavaju kvanten informacije koja se nalazi u fotonu.
Ovakva memorija zahteva koherentni sistem materije, jer bi u suprotnom kvantna informacija unitar uredjaja bila izgubljena zbog nekoherentnosti.
\subsection{Kvantno racunarstvo}
\end{document}
